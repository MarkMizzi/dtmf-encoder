\documentclass[11pt,a4paper]{scrartcl}

% Support for UTF-8 and non-English letters require the following two
\usepackage[T1]{fontenc}
\usepackage[utf8]{inputenc}
% Font packages
\usepackage[default,defaultsans,oldstyle,proportional]{lato}
\usepackage[scaled]{beramono}
\usepackage{sfmath}
\usepackage{fancyvrb}
% Optimized justification via improved microtypography on character level
\usepackage{microtype}
\usepackage{ragged2e}
\usepackage[none]{hyphenat} % disable all hyphenation
\setlength{\emergencystretch}{3em} % allow extra hfill, needed if hyphenation disabled
%\overfullrule=1mm % mark overfull boxes
%\usepackage{showframe} % show edges of text areas
% Page layout
\usepackage[left=20mm,right=20mm,top=20mm,bottom=20mm,
   nohead,foot=10mm]{geometry}
% SVN metadata
\usepackage[today,revrange,nofancy]{svninfo}
\svnInfo $Id$
% Page headers
\usepackage{scrlayer-scrpage}
\usepackage{lastpage}
% Page header
\KOMAoptions{headsepline=0pt,plainheadsepline=off}
\ihead*{}
\chead*{}
\ohead*{}
% Page footer
\KOMAoptions{footsepline=0pt,plainfootsepline=off}
\ifoot*{Last edited: \today}
\cfoot*{Page \thepage{} of \pageref{LastPage}}
\ofoot*{Document v.\svnInfoMaxRevision}
% Text layout
\KOMAoption{parskip}{never}
\newlength{\myparindent}
\newlength{\myparskip}
\setlength{\myparindent}{0pt}
\setlength{\myparskip}{5pt plus 1pt}
\setparsizes{\myparindent}{\myparskip}{0.1\linewidth plus 1fil}
\RedeclareSectionCommand[beforeskip=6pt,afterskip=3pt,afterindent=false]{section}
\RedeclareSectionCommand[beforeskip=6pt,afterskip=-0.5em,afterindent=false]{paragraph}
\RaggedRight

% Meeting details
\title{Meeting Minutes -- Group 1}
\author{Location: 0 B 6}
\date{27 February 2023, 11:00--13:40}

\begin{document}

\maketitle

\section*{Present}
% Members actually present at the meeting
Gabriel Apap,
Mark Mizzi,
Damjan Filipovic

\section*{Discussion}

\begin{enumerate}

% There should be an item corresponding to each of the agenda items for this
% meeting.

% Add other items here as necessary

\item Decide on allocation of minuting and agendas for next meetings.
   \begin{enumerate}
   \item Decided that since we will not have many meetings a solution that splits this the most equally is ideal.
   \item Mark proposes a solution. Gabriel does not understand. Mark draws diagram to aid explanation. Agreement on Mark's solution.
   \item First week Mark - Chair, Gabriel - Minutes. Second week - Chair - Damjan, Mark - Minutes. Third week - Chair - Gabriel, Minutes- Damjan etc
   \end{enumerate}

\item Go over the project requirements as specified by "customer" (in the handbook).
   \begin{enumerate}
      \subsection{Client Requirements}
      \item Clarification was made that the project requires to build a DTMF encoder only, decoding is not necessary
      \item It was confirmed that the experiment board contains a DAC. The board is equipped with a 10 Bit DAC.
      \item The meaning of "prioritised" was interpreted to mean the order of input as determined by the board. Although a human might think they were pressed at the same time
      however the board can determine exactly which key was pressed first.
      \item Clarification was made on how queued imput and output is handled. Gabriel misunderstood that an enter key was needed and that all output is handled at once.\\
      This is not the case, (as Mark explained) input is processed immediately. An output queue is used due to output most likely taking longer than input processing.
      \item Clarification was made by Gabriel on what symbol length and inter symbol spacing means. Symbol length is how long the dualtone is outputted for for each symbol
      inter symbol spacing being the time between each symbol output. For example a symbol out put of 250ms with 50ms space between each symbol.
      \subsection{Functional Requirements}
      \item Gabriel noted that point F.2 and F.3 are somewhat unclear given that an LCD is itself a digital ouput. Is any other digital output required?\\
      \textbf{To clarify with Prof Briffa next week (given this is relatively low priority).}
      \item Clarification was made on point F.7 by Mark on what possible errors could arise. Possible errors included disconnection of peripherals or the application of invalid settings.\\
      \textbf{To further clarify with Prof Briffa next week.}
      \subsection{Technical Requirements}
      \item Mark strongly suggests that the use of C++ should be avoided at all costs due to issues with dynamic linking and incomplete support of certain features.
      He further argues that we get no benefit from using C++. He notes that the use of libraries would not be available.\\
      The team decides to take Mark's suggestion. \textbf{The team will only be developing in C.}
      \item Clarification is needed on what is meant in T.6 by "multi-threaded" given that the CPU is a single core with no hardware threads. Mark posits that the
      meaning if multi-threaded in this context is the interleaving of interrupt code with regular code as well as the scheduling of regular code.\\
      \textbf{To clarify with Prof Briffa next week.}
      \subsection{Organisational Requirements}
      \item Mark reccommends the use of snake\_case for functions and variable names, and PascalCase for structs. It was decided that all curly brackets will be written on seperate lines.
      For switch statements, any case code must be written underneath the case declaration. Same line multi variable declaration to be allowed but used with best judgement.
      Any further code style decisions will be made in future when necessary.
      \item Commit messages should have a singular line explaining briefly what was changed. Any further clarifications to be made after that.
      \item \textbf{The team to set up DoxyGen by next week and briefly go over documentation for it.}
   \end{enumerate}

\item Discuss architectural overview of the design (graphical representation).
\begin{enumerate}
   \subsection{Interrupts}
   \item Keypad input interrupt handler. Mark clarifies a seperate interrupt handler will be needed for each symbol/key. Interrupt handler will push to an output queue to be handled by regular code.
   \item Potentitally the LCD, \textbf{research must be carried out on how the LCD functions.}
   \subsection{Regular code}
   \item Task that processes one output from the output queue. This task will have to write to LCD and produce the tone.
   \item Access to the hardware will be abstracted using "driver code". Example \verb!write_to_lcd("25")! \verb!output_tone(8)!
   \item Conversion code from numbers to strings
\end{enumerate}

\item Allocate research tasks for the team.
\begin{enumerate}
\item Gabriel - Research how the LCD functions. Look into amplifier circuit. Look into how DAC functions.
\item Mark - To look at optimal numbering of symbols to use bitwise tricks. To look into interrupt calling convention. To look at assembly instruction set and build up/find a quick reference (like a cheatsheet).
\item Damjan - To look into optimal conversion code for number to string. To look into thread safe and atomic data structures.
\end{enumerate}

\item Other matters
\begin{enumerate}
\item Next meeting to be held on the 2nd March
\end{enumerate}

\end{enumerate}

\section*{Actions}

\begin{enumerate}

\item Make necessary clarifications with Prof Briffa
\begin{flushright}
Assigned to: \textbf{Team} \\
Deadline: \textbf{6\textsuperscript{th} March}
\end{flushright}

\item Research
\begin{flushright}
Assigned to: \textbf{Mark} \\
Deadline: \textbf{next meeting}
\end{flushright}

\item Research
\begin{flushright}
Assigned to: \textbf{Damjan} \\
Deadline: \textbf{next meeting}
\end{flushright}

\item Research
\begin{flushright}
Assigned to: \textbf{Gabriel} \\
Deadline: \textbf{next meeting}
\end{flushright}

% Add more actions as necessary

\end{enumerate}

\end{document}
