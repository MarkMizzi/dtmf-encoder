\documentclass[11pt,a4paper]{scrartcl}

% Support for UTF-8 and non-English letters require the following two
\usepackage[T1]{fontenc}
\usepackage[utf8]{inputenc}
% Font packages
\usepackage[default,defaultsans,oldstyle,proportional]{lato}
\usepackage[scaled]{beramono}
\usepackage{sfmath}
% Optimized justification via improved microtypography on character level
\usepackage{microtype}
\usepackage{ragged2e}
\usepackage[none]{hyphenat} % disable all hyphenation
\setlength{\emergencystretch}{3em} % allow extra hfill, needed if hyphenation disabled
%\overfullrule=1mm % mark overfull boxes
%\usepackage{showframe} % show edges of text areas
% Page layout
\usepackage[left=20mm,right=20mm,top=20mm,bottom=20mm,
   nohead,foot=10mm]{geometry}
% SVN metadata
\usepackage[today,revrange,nofancy]{svninfo}
\svnInfo $Id: minutes_template.tex 6594 2023-02-22 15:50:48Z jbri2 $
% Page headers
\usepackage{scrlayer-scrpage}
\usepackage{lastpage}
% Code highlighting
\usepackage{minted}

% Page header
\KOMAoptions{headsepline=0pt,plainheadsepline=off}
\ihead*{}
\chead*{}
\ohead*{}
% Page footer
\KOMAoptions{footsepline=0pt,plainfootsepline=off}
\ifoot*{Last edited: \today}
\cfoot*{Page \thepage{} of \pageref{LastPage}}
\ofoot*{Document v.\svnInfoMaxRevision}
% Text layout
\KOMAoption{parskip}{never}
\newlength{\myparindent}
\newlength{\myparskip}
\setlength{\myparindent}{0pt}
\setlength{\myparskip}{5pt plus 1pt}
\setparsizes{\myparindent}{\myparskip}{0.1\linewidth plus 1fil}
\RedeclareSectionCommand[beforeskip=6pt,afterskip=3pt,afterindent=false]{section}
\RedeclareSectionCommand[beforeskip=6pt,afterskip=-0.5em,afterindent=false]{paragraph}
\RaggedRight

% Meeting details
\title{Meeting Minutes -- Group 5}
\author{Location: 0 B 6}
\date{1 March 2023, 14:30--17:30}

\begin{document}

\maketitle

\section*{Present}
% Members actually present at the meeting
Mark Mizzi,
Damjan Filipovic (Chair),
Gabriel Apap

\section*{Discussion}

\begin{enumerate}

% There should be an item corresponding to each of the agenda items for this
% meeting.

\item Minutes from the previous meeting were read and approved.

\item Matters arising out of minutes
    Some issues were brought up:
    \begin{enumerate}
        \item Some issues were brought up.
        \item Mark points out that his suggestion on avoiding C++ because of dynamic linking is incorrect as dynamic linking
            is not available on a bare metal platform.
        \item Gabriel points out that the statement made on having one interrupt handler per keypad button is incorrect since the
            keypad circuit is a diode matrix with a total of 8 usable pins.
    \end{enumerate}

\item Progress report from group members
   \begin{enumerate}
        \item Gabriel researched how the LCD and the DAC functions. He has not yet looked into the amplifier circuit,
            but this is expected to be straightforward.
        \item Damjan has looked into lock-free queues and thinks it is possible to implement them thanks to the
            hardware support offered by the Cortex M4. He has not looked into the details yet due to shortage of time.
        \item Mark looked into the instruction set and found a good cheat sheet for reference. He also decided that an
            optimal numbering would be to give keys in the same row a sequentially ordered numbering, as then the row
            and column of a key can be extracted using masks and shifts.
            He did not have time to look at interrupts in detail.
   \end{enumerate}

% Add other items here as necessary

\item Compare and contrast 2 high level designs for our system suggested by Gabriel and Mark.
Based on these results, a final decision for the implementation.

\begin{enumerate}
    \item The problem with our previous design discussion is that we did not consider that the keypad is a diode matrix,
        and has to be driven using polling.
    \item Gabriel suggests setting columns high, and detecting a voltage change in any of the rows. Such a voltage change
        triggers an interrupt, which polls the keypad to figure out which key was pressed.
        Non-interrupt code processes the output queue of items generated from interrupts.
    \item Mark suggests an alternate design where timer interrupts are used to trigger polling of the keypad.
        This ensures that the keypad is serviced at a known frequency.
        Non-interrupt code processes the output queue of items as before.
    \item Mark suggests using a Gantt chart to compare the two designs.
    \item Mark points out that an important design consideration is that queues must be emptied at least as fast as they are
        filled over some period of time otherwise the queue will overflow. Overflow is unlikely due to speed of human input.
    \item Gabriel comes up with an alternate design that does not involve a queue.
        His idea is to process output immediately inside the polling cycle, and
        then return to the polling cycle. If the output code section is fast enough, 
        any multi-key presses will be detected. This can be done inside an interrupt handler or we can just have the
        polling cycle as the regular code.
    \item A discussion ensued where the merits of using interrupt handlers vs having polling cycle as regular code were
        compared. There is not much of an engineering difference between the two choices (much of the code is the same),
        but maybe having the code inside an interrupt handler is more future proof, since extra background tasks can then
        be added as regular code.
    \item A decision was made to have \textbf{the polling cycle inside an interrupt handler} to allow us to add any 
        background tasks as needed in the future. It was later decided that the initial version of the system will not
        work like this (see below).
    \item It was pointed out that Gabriel's idea requires the use of a timer interrupt to disable the keypad-driven interrupt
        for some time after the last polling loop. (Since output processing happens faster than human input, not disabling the
        keypad interrupt for a certain amount of time would cause spurious output.)
    \item Mark and Damjan point out that while Gabriel's solution is the most sophisticated (best use of resources),
        it is also the most complex to implement. They suggest developing the system in three separate steps:
        \begin{enumerate}
            \item Having a system without interrupts that executes the polling cycle repeatedly in regular code.
                Busy waiting between polling cycles to avoid spurious output (this code will be scrapped).
            \item Polling loop executes on timer interrupts. Output is processed during a polling loop.
                 Regular code does nothing.
            \item Polling loop executes on keypad-driven interrupts. Timer interrupt is used to disable keypad-driven
                interrupt for a certain period of time after last run of polling loop (to avoid spurious output).
                Regular code does nothing.
        \end{enumerate}
        Each of these versions of the system can reuse most of the code from the previous version, while adding 
        more sophistication.
    \item It was decided to \textbf{develop the system in these 3 steps}.
\end{enumerate}

\item Discuss some core algorithms for the system.

    \subsection{Production of the DTMF tones in software}
    \begin{enumerate}
        \item Initially the team decided to sample a single sine wave in a lookup table with 256 entries, and then 
            set a fixed sampling period. Then to compute an approximation for a sine of period $T$ (frequency $f$) at time $t$,
            we average values from the LUT according to the following formula

            $$ \sin(2\pi ft) \approx \frac{1}{2}\Bigg(LUT\bigg[\frac{t}{T} \times 256\bigg] + LUT\bigg[-(-\frac{t}{T} \times 256)\bigg]\Bigg) $$

            Note the use of the equality 
            $$ -\bigg\lfloor - \frac{n}{m} \bigg\rfloor = \bigg\lceil \frac{n}{m} \bigg\rceil $$
            to get the two entries from the LUT needed.
        
        \item The problem with this approach is that we are given frequencies not periods. Converting these to periods
            will give us floating point numbers.

            Using frequencies instead of periods in the formula above does not solve this issue as then the index
            computation would involve $2\pi$:
        
            $$ \sin(2\pi ft) \approx \frac{1}{2}\Bigg(LUT\bigg[\frac{ft}{2\pi} \times 256\bigg] + LUT\bigg[-(-\frac{ft}{2\pi} \times 256)\bigg]\Bigg) $$

        \item Gabriel suggested using one sine wave's frequency $f$ to determine the sampling frequency as:
            $$ f_s = \frac{f}{256} $$

            In this way, the value of this sine wave at a particular sample number $i$ is just a simple lookup in the LUT (modulo $256$).

            The value of the other sine wave (frequency $f'$) at the same sample number $i$ can be computed using the formula:
            $$ \frac{1}{2}\Bigg(LUT\bigg[rem\bigg(\frac{f'i}{f}, 256\bigg)\bigg] + LUT\bigg[rem\bigg(-\bigg(-\frac{f'i}{f}\bigg), 256\bigg)\bigg]\Bigg) $$
            
            Note that the remainder after dividing by $256$ can be computed efficiently by masking with $0xff$.

        \item One drawback of this approach is that the sampling frequency is not constant. However by selecting the 
            higher frequency tone as the base, we can assure high quality sound output (sampling frequency in the range $309.5-418kHz$),
            which is below the DAC's maximum update frequency of $1 MHz$ but still good quality.

        \item Since the divisor is one of a set frequency of 4, the division in the formula above can actually be
            optimized away using magic numbers and shifts. This can be achieved using an additional LUT.

        \item Gabriel pointed out that a LUT of size 256 is overkill, since the worst sampling frequency in this case is 
            almost 10 times higher than $44.1\,\,kHz$ audio. A LUT of size 32 will do. (Same formulae).

            This gives us a sampling frequency of at least $1209 \times 32 \approx 38.6\,\,kHz$, definitely good enough.

        \item Some C code was written to illustrate the idea, and the produced assembly was viewed using Compiler Explorer.

            \begin{minted}{c}
int sin_lut[] = {
    2048,2447,2831,3185,3495,3750,3939,4056,
    4095,4056,3939,3750,3495,3185,2831,2447,
    2048,1648,1264,910,600,345,156,39,
    0,39,156,345,600,910,1264,1648,
};

#define SIN(baseidx, freq, basefreq) \
    ((sin_lut[(baseidx * freq / basefreq) & 0x1f] + \
      sin_lut[(-(-baseidx * freq / basefreq)) & 0x1f]) / 2)

// basefreq is the higher frequency
int sin_add(int basefreq, int freq, int idx) {
    return sin_lut[idx & 0x1f] + SIN(idx, freq, basefreq);
}
            \end{minted}
    \end{enumerate}

\item Other matters
   \begin{enumerate}
   \item It was noted that a background task is still needed to drive the DAC, since the symbol spacing can be up to 10s (i.e. the tone can be produced for 10s).
   \item This means an output queue is still required, but only for producing the tones (LCD output and computing the tone can be handled immediately).
   \end{enumerate}

\end{enumerate}

\section*{Actions}

\begin{enumerate}

\item Further research into the LCD, and conversion from strings to LCD encoding.
\begin{flushright}
Assigned to: \textbf{Gabriel Apap} \\
Deadline: \textbf{next meeting}
\end{flushright}

\item Research lock-free queues.
\begin{flushright}
Assigned to: \textbf{Damjan Filipovic} \\
Deadline: \textbf{next meeting}
\end{flushright}

\item Research into interrupt handlers, and looking further at assembly instructions.
\begin{flushright}
Assigned to: \textbf{Mark Mizzi} \\
Deadline: \textbf{next meeting}
\end{flushright}

% Add more actions as necessary

\end{enumerate}

\end{document}
